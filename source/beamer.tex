\documentclass{beamer}
\usepackage{tikz}
\usepackage[utf8]{inputenc}
\usepackage{beamerthemesplit}
\usepackage[english]{babel}
\usepackage{rotating}
\usepackage{amsmath,amsfonts,amssymb,amsthm}
\usepackage{latexsym}
\usepackage{enumerate}
\usepackage{grffile}
\usepackage{listings}
\usepackage{rotating}
\usepackage{color}
\usepackage{wasysym}
\usepackage{algorithm2e}
\usepackage{marvosym}
\usepackage{multicol}
\usetikzlibrary{calc,shapes.callouts,shapes.arrows,decorations.pathmorphing}

\title{How to create a simple Ansible module}
\author{Felix Fontein}
\date{June 27, 2019}

\usetheme{Frankfurt}
\setbeamertemplate{navigation symbols}{}

%\setbeamertemplate{background canvas}[vertical shading][bottom=red!10,top=blue!10]
%\usetheme{Warsaw}
%\usefonttheme[onlysmall]{structurebold}

%\newtheorem{definition}{Definition}[section]
%\newtheorem{theorem}[definition]{Theorem}
%\newtheorem{proposition}[definition]{Proposition}
%\newtheorem{lemma}[definition]{Lemma}
%\newtheorem{corollary}[definition]{Corollary}
%\newtheorem{remark}[definition]{Remark}
%\newtheorem{remarks}[definition]{Remarks}
%\newtheorem{example}[definition]{Example}
%\newtheorem{examples}[definition]{Examples}

\newcommand{\N}{\mathbb{N}}
\newcommand{\Z}{\mathbb{Z}}
\newcommand{\Q}{\mathbb{Q}}
\newcommand{\R}{\mathbb{R}}
\newcommand{\C}{\mathbb{C}}
\newcommand{\F}{\mathbb{F}}

\newcommand{\abs}[1]{{\left|{#1}\right|}}
\newcommand{\norm}[1]{{\left\|{#1}\right\|}}
\newcommand{\gen}[1]{{\left\langle{#1}\right\rangle}}
\newcommand{\floor}[1]{{\left\lfloor{#1}\right\rfloor}}
\newcommand{\Matrix}[1]{{\left(\begin{matrix} #1 \end{matrix}\right)}}

\DeclareFontShape{OT1}{cmtt}{bx}{n}{<5><6><7><8><9><10><10.95><12><14.4><17.28><20.74><24.88>cmttb10}{}

\setbeamerfont{itemize/enumerate body}{size=\Large}
\setbeamerfont{itemize/enumerate subbody}{size=\Large}
\setbeamerfont{itemize/enumerate subsubbody}{size=\footnotesize}

\definecolor{darkgreen}{rgb}{0,0.6,0}
\definecolor{darkred}{rgb}{0.6,0,0}
\definecolor{darkblue}{rgb}{0,0,0.6}
\definecolor{darkyellow}{rgb}{0.6,0.6,0}
\definecolor{lightred}{rgb}{1,0.5,0.5}
\definecolor{lightgreen}{rgb}{0.5,1,0.5}
\definecolor{lightblue}{rgb}{0.5,0.5,1}
\definecolor{lightyellow}{rgb}{1,1,0.5}
\definecolor{lightgray}{gray}{0.75}
\definecolor{verylightgray}{gray}{0.95}

\newcount\fadein
\newcount\fadeout
\newcount\colorfade
\newdimen\shift

\definecolor{mygreen}{rgb}{0,0.6,0}
\definecolor{mygray}{rgb}{0.5,0.5,0.5}
\definecolor{mymauve}{rgb}{0.58,0,0.82}
\lstset{language=python,
        basicstyle=\large\tt,
        commentstyle=\color{mygreen},
        keywordstyle=\color{blue},
        stringstyle=\color{mymauve},
        numbers=left,
        numberstyle=\tiny\color{mygray},
        stepnumber=1,
        numbersep=5pt}

\begin{document}

  \frame{\titlepage}

  \begin{frame}
    \frametitle{Roadmap}
    \tableofcontents%[pausesections]
  \end{frame}

  \section{Plugins and modules}
  \begin{frame}
    \frametitle{Action plugins and modules}
    \begin{itemize}
      \item \alert{Action plugins} behave like module
      \item They are executed on controller node
      \item They can transfer files to/from remote node
      \item They can call module

      \vspace{0.5cm}

      \item<2-> \alert{Modules} are executed on remote node
      \item<2-> Modules are copied and templated arguments are injected
      \item<2-> Modules cannot access controller's state
    \end{itemize}
  \end{frame}

  \begin{frame}
    \frametitle{Modules}
    \begin{itemize}
      \item Modules are executable files
      \item They receive input in JSON format
      \item They have to output JSON

      \vspace{0.5cm}

      \item<2-> Ansible has a lot of Python support code to make writing modules easy

      \vspace{0.5cm}

      \item<3-> That's what we want to use today!
    \end{itemize}
  \end{frame}

  \section{General principles}
  \begin{frame}
    \frametitle{General principles}
    \begin{itemize}
      \item UNIX philosophy: do only one thing, but do it well!
      \item<2-> Modules should be idempotent if somehow possible
      \item<3-> Different types of modules:
      \begin{enumerate}
        \item Modules which do something;
        \item \lstinline|_info| / \lstinline|_facts| modules
      \end{enumerate}
      \item<4-> You can find \emph{a lot} of material on \url{https://docs.ansible.com/ansible/latest/dev_guide/index.html}
    \end{itemize}
  \end{frame}

  \section{A simple module}

  \begin{frame}[fragile]
    \frametitle{A very simple module}
\begin{lstlisting}{}
#!/usr/bin/python
from ansible.module_utils.basic import (
  AnsibleModule
)

def main():
  module = AnsibleModule(dict())
  module.exit_json(msg="Hello!")

if __name__ == '__main__':
  main()
\end{lstlisting}
  \end{frame}

  \subsection{Running the module}
  \begin{frame}[fragile]
    \frametitle{Running the module}
    \begin{enumerate}
      \item Create test playbook

\begin{lstlisting}[language=bash, numbers=none]
$ ${EDITOR} example01.yaml
\end{lstlisting}
      \item Create subdirectory «library»

\begin{lstlisting}[language=bash, numbers=none]
$ mkdir library
\end{lstlisting}
      \item Put module into there

\begin{lstlisting}[language=bash, numbers=none]
$ ${EDITOR} library/example01.py
\end{lstlisting}
      \item Execute playbook!

\begin{lstlisting}[language=bash, numbers=none]
$ ansible-playbook example01.yaml
\end{lstlisting}
    \end{enumerate}
  \end{frame}

  \subsection{Arguments}
  \begin{frame}[fragile]
    \frametitle{A simple module with arguments}
\begin{lstlisting}{}
def main():
  argument_spec = dict(
    name=dict(type='str',
              required=True),
    some_number=dict(type='int'),
    colors=dict(type='list',
                elements='str'),
    state=dict(type='str',
               default='present',
               choices=['present', 'absent']),
    password=dict(type='str',
                  no_log=True),
  )
  module = AnsibleModule(argument_spec)
  module.exit_json(**module.params)
\end{lstlisting}
  \end{frame}

  \begin{frame}[fragile]
    \frametitle{A simple module which can fail}
\begin{lstlisting}{}
import os

def main():
  argument_spec = dict(
    file=dict(type='path',
              required=True),
  )
  module = AnsibleModule(argument_spec)

  fn = module.params['file']
  if not os.path.exists(fn):
    module.fail_json(
      msg='"%s" does not exist!' % fn
    )
  module.exit_json(msg='"%s" exists' % fn)
\end{lstlisting}
  \end{frame}

  \begin{frame}[fragile]
    \frametitle{A simple module which runs a command}
\begin{lstlisting}{}
def main():
  module = AnsibleModule(dict())

  rc, stdout, stderr = module.run_command(
    ['ls', '/'],
    check_rc=True
  )
  # Ansible will fail if rc is != 0

  files = [
    f for f in stdout.split('\n') if f
  ]
  module.exit_json(files=files)
\end{lstlisting}
  \end{frame}

  \begin{frame}[fragile]
    \frametitle{A simple module which POSTs to a URL}
\begin{lstlisting}{}
from ansible.module_utils.urls import \
                                 fetch_url
def main():
  module = AnsibleModule(dict())
  r, i = fetch_url(
    module, 'https://httpbin.org/',
    method='POST', data='x=42'
  )
  status = i["status"]
  try:
    body = r.read()
  except:
    body = i.get('body')
  body = module.from_json(body)
  module.exit_json(status=status, body=body)
\end{lstlisting}
  \end{frame}

  \subsection{Debugging}
  \begin{frame}[fragile]
    \frametitle{What if something goes wrong?}
    \begin{enumerate}
      \item If module crashes, \lstinline{ansible-playbook -vvv}
      will provide stacktrace
      \item<2-> \lstinline{print()} will not work
      \item<3-> Poor man's debugging:
\begin{lstlisting}[numbers=none]
module.exit_json(msg='...')
raise Exception('xxx')
\end{lstlisting}
      \item<4-> Use \lstinline{q} (get via \lstinline{pip install q}):
\begin{lstlisting}[numbers=none]
import q
q.q(value)
\end{lstlisting}
      Meanwhile, do \lstinline{tail -f /tmp/q}
    \end{enumerate}
  \end{frame}

  \section{Documentation}
  \begin{frame}[fragile]
    \frametitle{Documenting modules}
    \begin{itemize}
      \item Documentation is included as Python strings
      \item Keys \lstinline{DOCUMENTATION} and \lstinline{RETURN} are YAML
      \item Key \lstinline{EXAMPLES} is essentially a list of Ansible tasks

      \vspace{0.5cm}
      \item<2-> \lstinline{DOCUMENTATION} contains module name, description, requirements, author, and parameter documentation
      \item<2-> \lstinline{RETURN} documents returned variables
      \item<2-> \lstinline{EXAMPLES} gives examples how to use the module
    \end{itemize}
  \end{frame}

  \begin{frame}[fragile]
    \frametitle{Header}
\begin{lstlisting}
#!/usr/bin/python
#
# Copyright 2019 Felix Fontein
# GNU General Public License v3.0+ (see COPYING
# or https://www.gnu.org/licenses/gpl-3.0.txt)

from __future__ import absolute_import, ...
__metaclass__ = type

ANSIBLE_METADATA = {
  'metadata_version': '1.1',
  'status': ['preview'],
  'supported_by': 'community'
}
\end{lstlisting}
  \end{frame}

  \begin{frame}[fragile]
    \frametitle{\lstinline{DOCUMENTATION} 1/2}
\begin{lstlisting}[showstringspaces=false]
DOCUMENTATION = r'''
---
module: example06
short_description: Makes a POST
description:
  - Makes a POST to a URL.
  - This is really only a test.
version_added: "2.9"
author:
  - "Felix Fontein (@felixfontein)"
requirements:
  - mycustommodule >= 1.0
  - An internet connection
\end{lstlisting}
  \end{frame}

  \begin{frame}[fragile]
    \frametitle{\lstinline{DOCUMENTATION} 2/2}
\begin{lstlisting}[showstringspaces=false]
DOCUMENTATION = r'''
---
[...]
options:
  url:
    description:
      - The URL to POST to
    required: yes
    type: str
'''
\end{lstlisting}
  \end{frame}

  \begin{frame}[fragile]
    \frametitle{\lstinline{EXAMPLES}}
\begin{lstlisting}[showstringspaces=false]
EXAMPLES = r'''
- name: Do something
  example06:
    url: https://example.com
  register: result
- debug:
    msg: |
      {{ result.status }} --
      {{ result.body }}

# This must be valid YAML. Correct Ansible
# syntax is not checked, better do that
# yourself!
'''
\end{lstlisting}
  \end{frame}

  \begin{frame}[fragile]
    \frametitle{\lstinline{RETURN}}
\begin{lstlisting}[showstringspaces=false]
RETURN = r'''
status:
  description:
    - The status code of the POST request
  returned: always
  type: int
  sample: 200
body:
  description:
    - The parsed JSON result
  returned: always
  type: dict
  sample: '{"key": "value"}'
'''
\end{lstlisting}
  \end{frame}

  \begin{frame}[fragile]
    \frametitle{Viewing the documentation}
    \begin{enumerate}
      \item Using \lstinline{ansible-doc}:
\begin{lstlisting}[language=bash, numbers=none]
$ export ANSIBLE_LIBRARY=library
$ ansible-doc example06
\end{lstlisting}
      \item<2-> Build HTML documentation with Sphinx
      \begin{itemize}
        \item Clone \url{git@github.com:ansible/ansible.git}
        \item Put your module into \lstinline{lib/ansible/modules}
        \item In \lstinline{docs/docsite} run \lstinline{MODULES=example06 make webdocs} and wait
        \item Open \lstinline{_build/html/modules/example06_module.html}
      \end{itemize}
    \end{enumerate}
  \end{frame}

  \section{Testing}
  \begin{frame}
    \frametitle{Testing}
    \begin{itemize}
      \item Sanity tests
      \begin{itemize}
        \item static code analysis (flake8, ...)
        \item validate argument spec, compare to documentation
        \item try to build documentation
      \end{itemize}
      \item Unit tests
      \item Integration tests
      \begin{itemize}
        \item Essentially Ansible roles
      \end{itemize}
    \end{itemize}
  \end{frame}

  \begin{frame}[fragile]
    \frametitle{Sanity Testing with ansible-test}
    \begin{itemize}
      \item Assume you have your module embedded in the Ansible GIT tree
      \item Module: \lstinline{lib/ansible/modules/example06.py}
    \end{itemize}

\begin{lstlisting}[language=bash, numbers=none]
test/runner/ansible-test sanity \
    lib/ansible/modules/example06.py \
    --docker-no-pull --docker
\end{lstlisting}
  \end{frame}

  \begin{frame}[fragile]
    \frametitle{Integration Testing with ansible-test}
    \begin{itemize}
      \item Module: \lstinline{lib/ansible/modules/example06.py}
      \item Tests: \lstinline{test/integration/targets/example06/}
      \begin{itemize}
        \item \lstinline{aliases}: CI options, other modules which have the same tests
        \item \lstinline{meta/main.yml}: dependencies (like \lstinline{prepare_http_tests})
        \item \lstinline{tasks/main.yml}: the tests themselves
      \end{itemize}
    \end{itemize}

\begin{lstlisting}[language=bash, numbers=none]
test/runner/ansible-test integration -v \
    --docker-no-pull --docker ubuntu1804 \
    example06
\end{lstlisting}
  \end{frame}

  \section{Getting Your Module into Ansible}
  \begin{frame}
    \frametitle{Getting Your Module into Ansible (1/2)}
    \begin{itemize}
      \item \alert{Really read (parts of) the dev guide!}
      \url{https://docs.ansible.com/ansible/latest/dev_guide/index.html}
      \item In particular: \href{https://docs.ansible.com/ansible/latest/dev_guide/developing_modules_checklist.html}{\alert{«Contributing your module to Ansible»}} checklist
      (search for «I want to contribute my module or plugin.»)
      \item Definitely \alert{add} integration \alert{tests}, even if marked
      as \lstinline{unsupported}
    \end{itemize}
  \end{frame}

  \begin{frame}
    \frametitle{Getting Your Module into Ansible (2/2)}
    \begin{itemize}
      \item Make sure sanity and integration tests work before pushing
      \item Push to GitHub, create PR
      \item Make sure to fill in the provided form
      \item Make sure that CI (Shippable) runs through
      \item Tell people about your PR!
      \item<2-> Now you need \alert{reviews}...
    \end{itemize}
  \end{frame}

  \begin{frame}[fragile]
    \frametitle{Getting Your New Module PR Reviewed}
    \begin{itemize}
      \item Ask people you know who could use the module to test it
      \item All kind of user feedback is welcome
        \begin{itemize}
          \item how it was to use
          \item strange/unexpected behavior, missing functionality
          \item bugs
        \end{itemize}
      \item<2-> Try to get people to review the code
      \item<2-> Also try to get people to review all the Ansible details
      \item<2-> Ask nicely in \verb|#ansible-devel| (FreeNode)
    \end{itemize}
  \end{frame}

  \section*{}
  \begin{frame}
    \begin{center}
      \Huge Thank You\\
      for your attention!

      \vspace{1.5cm}

      \LARGE Questions? Comments?
    \end{center}
  \end{frame}
\end{document}
